\section{The World}

Each game of {\it Shadow of the Wyrm} starts off in one of two places: 
either on Little Island, a tiny piece of land at the edge of the world;
or in Carcassia, an island in the centre of the world.

On the north shore of Little Island is a small fishing village, the only
outpost on the island.  This is the village Isen Dun (literally 
``Iron Hill''), a fishing outpost founded hundreds of years ago.  Its
inhabitants are human, members of a tribe called the Iselings.  At the 
north-western edge of the island is an old set of ruins, while an ancient
stone stair sits at the southern tip.

On the south-east of the Isle of Carcassia is a city of the same name.  
This city is the seat of a great empire, one which extends its reach 
throughout the central lands, maintaining an uneasy peace with the other 
races and regions world.  It is a land of great opportunity and danger.

\subsection{Starting Out}

Isen Dun is full of friendly inhabitants, and is a great place to start
out.  The ploughman maintains a public garden in the village green for the
needy, and there are a number of NPCs (non-player characters) that have
quests that the player can try out.  But be careful --- some of these quests
are much more dangerous than others! 

Carcassia is a little more unfriendly.  It's a large city, and there are
many places to explore, opportunities for advancement.  But some of its
inhabitants are less than friendly, and wouldn't think twice about 
slitting a traveller's coinpurse or throat.  There are guilds and shops,
puzzles and dungeons: a great many things to learn.

\subsection{Non-Player Characters}

NPCs fall into two categories:

\begin{itemize}
\item Those that want to kill you.
\item Those that don't want to kill you.
\end{itemize}

Philosophers have argued this dichotomy for years, and the more Hobbesian
believe that the second category could simply be termed ``those that don't
want to kill you just yet.''  Debate rages.

Those that want to kill you should either be avoided, or destroyed,
depending on your preference and the general difficulty of the thing
bearing down on you.  You can get a general feel for creature difficulty
by using the look command (see Commands), and then hovering over another 
creature.  In addition to telling you what the creature is, the look 
command will give an indication of the creature's level, relative to yours.

\subsection{Finding Help}
\subsubsection{Adventurers}
The philosophers were right --- life is, on the whole, nasty, brutish and
short.  Most people are born into little or no means, and have to eke out an
existence through farming or some sort of trade.  Disease is rampant.
Economies shaky.  The more enterprising are always looking for a way to make
their lives a little easier.

Open thievery and robbery are illegal, and liable to get you killed if caught,
but there's always treasure hunting.  The world has been populated, it is
thought, for tens upon tens of thousands of years.  In that time, all kinds of
empires have fallen, treasures buried, entire peoples (and their priceless
artifacts) coming and going in the span of just a few hundred years.

This opens up a business opportunity.  The ownership of such things is typically
seen as falling to whomever finds it.  And so, in the taverns and social houses
around the world, people band together, agree to split whatever they find, and
set off to the nearest dungeon or cavern.  

These adventurers are eager, but green --- though they're happy to join you,
they're often little more than ordinary folk with a bit of equipment and 
training, and sometimes barely that.  If you're looking to join with people a 
bit more powerful, who've survived a few expeditions, you'll need to look 
elsewhere.

\subsubsection{Hirelings}
Throughout the world, you may run across hirelings who will fight with you ---
for a fee.  Seasoned adventurers who have made it through dungeons and caverns
by way of skill and luck, hirelings can be adventurers or warriors, wizards,
shepherds, or any class that you might pick yourself.  The amount that they
require is based on their own experience.

Each hireling wears a selection of weapons and armour that they've found or
enchanted in their travels.  That said, they're not averse to picking up and
wearing other items, as long as they're better than what they own.

Hirelings will act based on their circumstances and the situation, attacking
nearby threats, picking up and using interesting items, and otherwise following
their leader.  But to give specific orders to override these behaviours (stand
still, follow no matter what, etc), the order command, described in the 
Command Reference, can be used to do this.

\subsection{Combat}

The world is a dangerous place.  The outer lands are home to Snakelings,
Goblins, and Ogres.  Monsters, formerly the stuff of ancient myth, have
begun to appear once more.  And priests of the dark three stalk the world
like shadows, snatching the weak and unwary for blood sacrifices.  What's
an adventurer to do?

Combat is a popular option.

\subsubsection{Attack Difficulty and Speed}

Attacking uses the concept of attack difficulty to help determine whether
an attack hits, speed to determine when the creature acts next after the 
attack, and damage to indicate what kind of punishment the attack doles out.

\begin{itemize}
\item {\bf Base Difficulty} is the difficulty of the attack or weapon
without any statistic or skill modifiers applied.  A lower difficulty
indicates a better chance to hit.
\item {\bf Total Difficulty} is the difficulty of the attack or weapon
after all modifiers are applied.
\item {\bf Speed} is the speed of the attack.  A lower speed value
indicates a quicker attack.
\item {\bf Damage} is the range of the damage.  Damage can involve dice
and modifiers.  A dice roll of 1d6 is like rolling a six-sided die once,
meaning 1-6 damage can be dealt.  2d3+6 would mean that a 3-sided die
is rolled twice, with 6 added to the result.
\end{itemize}

\subsubsection{General Rules}

The basic internal rules for combat are:

\begin{enumerate}
\item Generate the ``target number'' --- a combination of the attacker's
weapon difficulty, weapon skill, target's evade, and other considerations.

\item Roll 1d100.
  \begin{itemize}
  \item If the roll is 5 or less, the attack automatically misses.

  \item Otherwise, if the d100 roll plus modifiers is greater than the 
  target number, the attack is considered a hit.  

  \item When the d100 roll is 96--99, the attack is a {\it mighty blow}.

  \item When the d100 roll is 100, the attack is a {\it critical hit}.
  \end{itemize}
\item Roll damage.
  \begin{itemize}
  \item For a regular hit, a random value is rolled using the attack's
  damage dice.

  \item For a mighty blow, the maximum damage is taken.

  \item For a critical hit, the maximum damage is taken.
  \end{itemize}

\item Consider resistances.  Multiply the damage by the creature's
resistance to that damage type.

\item If the attack is a slaying attack that affects the creature based on
its race, double the damage.  Typically only certain artifact weapons will
grant slaying against a race or family of races.

\item Subtract the defending creature's Soak score.  For a regular hit
or mighty blow, the full Soak is subtracted.  For a critical hit, only a
quarter of the Soak is subtracted.
\end{enumerate}

\subsubsection{Melee Combat}

Melee combat encompasses attacking barehanded, or with a hand-held
weapon.  After rolling the d100 to hit, bonuses are added based on the
creature's Combat score and skill in the given weapon.  Additionally, if
the weapon is particularly heavy, a bonus is added based on the creature's
Strength score.  If the weapon is not particularly heavy, or if the
creature is attacking without a weapon, a bonus is added based on the 
creature's Dexterity.  For light weapons, a penalty is applied based on
drunkenness.  This penalty is not applied for heavy weapons, where brute
strength, not finesse, powers the blow.

\subsubsection{Ranged Combat}

Ranged combat adds a bonus based on the creature's Archery skill, rather
than Combat, adds a bonus based on the weapon skill, and then adds a 
bonus based on Dexterity.  A penalty is applied based on drunkenness.

\subsubsection{Magical Combat}

Magical combat (e.g., attacking with a Wizard's or Witchling's spells)
adds bonuses based on the Magic skill and the creature's Intelligence
score.  Penalties are applied based on drunkenness and level of burden.

\subsection{Quests}

But not everything in the world wants to kill you.  Some NPCs are capable 
of granting quests.  Quests are unlocked by having the appropriate 
pre-requisites, and then talking to a particular NPC.  Quests will show up 
in the list of quests (see Commands).  When quests are completed, a reward 
is typically given --- an item, skill training, etc.  Quests typically 
provide rewards that help much more than the random items found on the 
dungeon floor, but tend to be dangerous, and should be taken with caution.

\subsection{The Passing of Time}

As your character moves through the world, time passes.  And as time
passes, the moon's phase changes as well.  The moon has long been known to
have a direct effect on the forces in the world - not only tidal, but the
elements, and primordial forces, too.  When the moon waxes or hangs full, 
elemental and celestial damage is more potent; when the moon wanes or
stands new, acid, poison, shadow, and arcane forces are at their peak.
The moon is also said to affect other things, such as the ability to train
one's strength or intelligence, but these forces are not well-understood.

