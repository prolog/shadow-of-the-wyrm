\section{The World}

Each game of {\it Savage Lands} starts off on Little Island, a tiny piece
of land at the edge of the world home to the people that populate Isen Dun
(literally ``Iron Hill''), a village perched atop a large hill.  At the
northern edge of the island is an old set of ruins, while an ancient stone
stair sits at the southern tip.

\subsection{Starting Out}

Isen Dun is full of friendly inhabitants, and is a great place to start
out.  The ploughman maintains a public garden in the village green for the
needy, and there are a number of NPCs (non-player characters) that have
quests that the player can try out.  But be careful --- some of these quests
are much more dangerous than others! 

\subsection{Non-Player Characters}

NPCs fall into two categories:

\begin{itemize}
\item Those that want to kill you.
\item Those that don't want to kill you.
\end{itemize}

Philosophers have argued this dichotomy for years, and the more Hobbesian
believe that the second category could simply be termed ``those that don't
want to kill you just yet.''  Debate rages.

Those that want to kill you should either be avoided, or destroyed,
depending on your preference and the general difficulty of the thing
bearing down on you.  You can get a general feel for creature difficulty
by using the look command (see Commands), and then hovering over another 
creature.  In addition to telling you what the creature is, the look 
command will give an indication of the creature's level, relative to yours.

\subsection{Combat}

The world is a dangerous place.  The outer lands are home to Snakelings,
Goblins, and Ogres.  Monsters, formerly the stuff of ancient myth, have
begun to appear once more.  And priests of the dark three stalk the world
like shadows, snatching the weak and unwary for blood sacrifices.  What's
an adventurer to do?

Combat is a popular option.

\subsubsection{General Rules}

The basic internal rules for combat are:

\begin{enumerate}
\item Generate the ``target number'' - a combination of the attacker's
weapon difficulty, weapon skill, target's evade, and other considerations.

\item Roll 1d100.
  \begin{itemize}
  \item If the roll is 5 or less, the attack automatically misses.

  \item Otherwise, if the d100 roll plus modifiers is greater than the 
  target number, the attack is considered a hit.  

  \item When the d100 roll is 96--99, the attack is a {\it mighty blow}.

  \item When the d100 roll is 100, the attack is a {\it critical hit}.
  \end{itemize}
\item Roll damage.
  \begin{itemize}
  \item For a regular hit, a random value is rolled using the attack's
  damage dice.

  \item For a mighty blow, the maximum damage is taken.

  \item For a critical hit, the maximum damage is taken.
  \end{itemize}

\item Consider resistances.  Multiply the damage by the creature's
resistance to that damage type.

\item If the attack is a slaying attack that affects the creature based on
its race, double the damage.  Typically only certain artifact weapons will
grant slaying against a race or family of races.

\item Subtract the defending creature's Soak score.  For a regular hit
or mighty blow, the full Soak is subtracted.  For a critical hit, only a
quarter of the Soak is subtracted.
\end{enumerate}

\subsubsection{Melee Combat}

Melee combat encompasses attacking barehanded, or with a hand-held
weapon.  After rolling the d100 to hit, bonuses are added based on the
creature's Combat score and skill in the given weapon.  Additionally, if
the weapon is particularly heavy, a bonus is added based on the creature's
Strength score.  If the weapon is not particularly heavy, or if the
creature is attacking without a weapon, a bonus is added based on the 
creature's Dexterity.

\subsubsection{Ranged Combat}

Ranged combat adds a bonus based on the creature's Archery skill, rather
than Combat, adds a bonus based on the weapon skill, and then adds a 
bonus based on Dexterity.

\subsubsection{Magical Combat}

Magical combat (e.g., attacking with a Wizard's or Witchling's spells)
adds bonuses based on the Magic skill and the creature's Intelligence
score.

\subsection{Quests}

But not everything in the world wants to kill you.  Some NPCs are capable 
of granting quests.  Quests are unlocked by having the appropriate 
pre-requisites, and then talking to a particular NPC.  Quests will show up 
in the list of quests (see Commands).  When quests are completed, a reward 
is typically given --- an item, skill training, etc.  Quests typically 
provide rewards that help much more than the random items found on the 
dungeon floor, but tend to be dangerous, and should be taken with caution.
