\documentclass{article}
\usepackage{longtable}

\begin{document}

\title{Guidebook for the Savage Lands}
\author{Julian Day}
\maketitle
\thispagestyle{empty}

\newpage
\clearpage
\thispagestyle{empty}

\tableofcontents

\newpage
\setcounter{page}{1}

\section{Introduction}

You wake up.  Your head throbs, and your clothes drip with saltwater. 
Around you lie a few pieces of driftwood, the sight of which triggers
a series of memories: a ship-journey from your homeland, full of promise;
storm clouds on the horizon, roiling and swirling; and the screams of the
sailors on the deck above as the storm caught the ship in its fury, tearing
it apart without mercy.

Somehow, you survived, clinging to several planks that miraculously
remained joined to form a small raft. And just as lucky, you still had all
your possessions, your backpack balanced precariously at the end of the
boards.  Drifting over the cold sea, you ate sparingly, and drank even
less.  Eventually, your water ran out, and, fighting a fever, you drifted
into unconsciousness, expecting to awake on the other side.

But it was not to be.  You stand on the rocky beaches of a small island,
your raft shattered against the rocks.  To the north-west, you can see the 
ruins of some sort of keep or fortification.  Closer to you, nestled to the
north, is a small village, smoke billowing from the short chimneys of the 
sod huts.  On the eastern outskirts, you can see a graveyard, and beyond 
that, a small orchard.  And at the south-eastern tip of the island, some 
sort of rocky outcropping holding out against the sea.
 
You left in search of adventure: monsters were returning, creatures of myth
and legend becoming startlingly real.  Your friends set sail a year
earlier, carrying their possessions on their backs, seeking their fame and
fortune.  You said your goodbyes, certain that, for one reason or another,
you were unlikely to see them again.

After that year, the wanderlust took you, too.  You took what you could
carry, selling the rest.  A small sailing ship offered passage for a 
handful of ivories, so long as you stayed out of the way.  You
jumped at the opportunity, and now you stand, dripping and exhausted, on the
shores of a place so small you don't recognize it from any map.

The world is huge, and this is but a small island.  But first things first:
you need to find a boat.

\section{Your Character}

Characters in {\it Savage Lands} are created by selecting a sex, race,
class, and deity.  A character's sex is purely for flavour, and has no 
effect on gameplay.  But a creature's race and class are much more 
important.  They determine starting statistics, skills, and equipment, and 
have a large impact on the difficulty of the game.  The choice of deity
will one day become important, but for now, all deities behave the same and
so it does not matter which one is selected.

\subsection{Races}

The races of {\it Savage Lands} range from diminutive to massive, and each
race has its own strengths and weaknesses.  Some make great warriors, 
while some are better suited for magic or thievery; there are others that
are more neutral and can work with any class.  The races of {\it Savage
Lands} are listed below.

\begin{itemize}
\item {\bf Humans} live throughout the world, in the centre lands, on
islands large and small, making their homes in cities and settlements, 
outposts and camps.  They are short and tall, fair and dark, settled and 
nomadic, but always adaptive, always resourceful, and always able to make 
the most of any situation.

\item {\bf Wood Elves} retreated into the forests millennia ago, building 
their civilizations in the treetops and branches.  Within the woods, they 
blend in with their surroundings, their clothing dyed deep shades of green 
and brown.  Wood Elves are skilled archers, and are known for their 
devotion to swordsmanship.

\item {\bf Mountain Elves} live high up in the mountains, far above the 
rest of the world.  They are stockier than other elves, and even many 
humans, fortified by a life in the thin, cold air.  Mountain Elves 
typically wear animal skins, and craft weapons out of the tools at hand, 
making use of stone, animal bones, and the little wood at hand.  They are 
skilled with clubs, hammers, slings, and stones.

\item {\bf Snakelings} are the product of black magic, wielded by some
unknown sorceror many thousands of years ago to fuse man and serpent. 
The resultant creatures horrified the world: standing nearly as tall as men,
the creatures have the scales and head of an enormous snake, with the 
remaining features human in appearance.  Cast to the fringes of the world, 
Snakelings band together in fens and marshes, hunting under darkness with 
barbed spears and javelins, generations of isolation filling their minds 
with a black and heavy hate.

\item {\bf Fae} are the remnants of the so-called faerie folk that lived in
the world far before the elves.  They require neither food nor drink, but 
enjoy both, holding great feasts deep within ancient forests.  Time has 
brought down the Fae, and few remain.  Those that do tend to travel 
together in caravans through the most inhospitable of terrain, keeping 
their company close and blocking out the rest of the world.  Fae are 
stealthy tricksters, and tremendously skilled with magic.

\item {\bf Dwarves} live deep within mountains, far below the 
naturally-occurring caverns and caves that serve as the entranceways.  They
carve their cities from the rock itself, and mine the mountains for its 
ore.  They are short, about two-thirds the size of men, but stocky, and 
disproportionately strong.  Dwarves are skilled with axes, hammers and 
crossbows.

\item {\bf Goblins} are short, cunning humanoids with sharp features and 
sharper teeth.  They live in small tribes on the outer islands, though 
recently they have been spotted closer and closer to the centre lands.   
Despite their smaller size, they are ferocious creatures, afraid of 
nothing.  They wear ratty, tattered clothing, and are skilled with their 
sharp, curved blades.

\item {\bf Ogres} are huge and thickly-built, with the strength of many 
men.  Though they are often seen with goblins, and share those creatures 
fierce nature, they are not nearly as intelligent.  Ogres make up for this 
with a tremendous strength and fortitude.  They are often seen wearing 
scraps of armour plundered from their kills, and prefer great clubs and 
hammers above all else.

\item {\bf Giants} are even older than the elves, and warred with the 
faerie folk when the world was still young.  Massive and tremendously 
strong, they stand twice as tall as the tallest man.  They live in the 
outer reaches of the world, where their settlements are constantly under 
attack by goblins and ogres, whom they hate. Giants prefer large, stone 
clubs, though when these are unavailable, a small tree will often do.

\item {\bf Gnomes} are tiny creatures that live in caverns and caves, and 
have since recorded time.  They stand even shorter than dwarves, and are 
much less strong.  They are intelligent and quick; gnomes love riddles and 
puzzles, and have a deep affinity for magic.  In combat, they prefer knives,
daggers, and small swords.
\end{itemize}

\subsection{Classes}

A character's class represents its background or occupation.  There are
many possibilities: powerful warriors, mighty wizards, holy pilgrims; but
also skilled smiths and artisans, street-hardened pugilists, and the
ubiquitous adventurer.  Class influences a character's starting statistics
and equipment, and grants its own set of initial skills.

\begin{itemize}
\item {\bf Adventurers} travel the world, seeking to make money and a name 
for themselves.  While they lack the raw strength of Warriors, the stealth 
and cunning of Thieves, or the great learning of Wizards, they nonetheless 
have characteristics of all three, and are well-prepared for the challenges
that lie ahead.

\item {\bf Apothecaries} prepare potions, salves, and remedies, assisting 
physicians by providing their medical materials.  This training gives 
apothecaries a practical knowledge of medicine and healing.  In addition, 
many dabble in hedge magic and cantrips, providing some practical magical 
knowledge while remaining focused on the duties of their trade.

\item {\bf Archers} are combatants skilled with ranged weapons such as 
bows, slings, and crossbows.  While others can also learn these weapons 
effectively, Archers can maintain their accuracy over much greater 
distances.  Though they possess some of the skill of Warriors in close 
combat, their focus is on ending the danger before it gets too close.

\item {\bf Artisans} are creative folk who craft jewellery, clothing, and 
other useful items.  With training, they can focus their efforts and craft 
truly spectacular creations.  Artisans learn their trade after apprenticing
to a master crafter for many years, and with their training complete, are
finally ready to make their way.

\item {\bf Merchants} travel from place to place, always looking to sell 
their wares.  In their travels, they learn a great many things about their 
merchandise, and are experts at identification.  Each merchant begins his 
or her travels with a small inventory of weapons, armour, rings, or staves.

\item {\bf Minstrels} are itinerant musicians.  They are instrumentalists 
and singers, learned in the rudiments of almost any form of music.  Others 
find them quite charismatic and charming.  Minstrels are often able to stir
and sway their audiences on the strength of a particularly good performance.

\item {\bf Nobles} are born of privilege and money.  Lords and Ladies, 
Kings and Queens, Thanes and Chiefs; all are examples of the fortunate 
nobility.  A life free from hard labour allows nobles to focus their 
efforts on leadership and military training, to better lead their people 
to victory.

\item {\bf Oracles} possess a keen sense of the future and present, and can
see things that others can't: happiness, true love, long life; but also 
hexes, illnesses, and death.  As favoured creatures of fate, they are 
immune to bad luck and curses.  Their predictions have an almost unfailing 
accuracy, which causes Oracles to be both respected and feared.

\item {\bf Pilgrims} are devout followers and messengers of the divine.  
They seek enlightenment through travel, devotion, and prayer.  They carry 
their life on their back, travelling from place to place, rarely staying 
long.  They are favoured by the Nine, who grant them the ability to learn
divine mysteries written down by monks and priests throughout the ages.

\item {\bf Pugilists} are fighters who focus solely on unarmed combat, 
relying on their fists and instincts to get them out of trouble.  
Quick-witted and nimble, they eschew the trappings of any armour heavy 
enough to hinder their movements.

\item {\bf Rovers} are solitary figures who spend the majority of their 
lives in the wild.  At home far from civilization, rovers can be found in 
the highest peaks, deepest forests, and darkest caves.  They spurn a life 
of comfort and prefer instead a life outdoors, surviving in the most 
inhospital parts of the world.

\item {\bf Sages} are scholars.  They study the seen and unseen to the 
exclusion of all else.  Sages are thin and frail.  Living alone, they often
withdraw from society to devote their lives to study.  As they increase 
their knowledge and edge closer to enlightenment, they gain access to the 
arcane and divine, the mystic and the primordial.

\item {\bf Seafarers} are drawn to the open water.  They make their living 
travelling the rivers and oceans, spurning a life on land.  The weather 
hardens them, and they are better able to withstand the rigors of cold.  
Fishers, pirates, and sailors are all examples of Seafarers.

\item {\bf Shepherds} are among the weakest and most lowly in society.  
Tending to their flocks, or those of others, they live at the fringes of 
civilization, ekeing out a living from the land.  They are not strong in 
combat, nor with magic, but some have said they enjoy a special status 
with the divine.

\item {\bf Smiths} are artisans of iron and steel.  Using their bellows, 
hammers, and anvils, they are able to improve upon weapons and armour, 
able to turn the most mediocre example into a masterpiece.  From spending 
a lifetime in front of the forge, they are bothered little by great heat.

\item {\bf Thieves} are nimble and fleet of foot.  They specialize in the 
redistribution of wealth, either to themselves, or to others.  Loosely 
organized into guilds, they learn early on how to pick pockets, open 
locks, scale walls, and deal with traps.

\item {\bf Warriors} include soldiers, barbarians, nomads, and sell-swords.
Some learn their skills by military training, while others learn simply 
through survival.  Coming from many walks of life, they all have skill in 
close combat.

\item {\bf Witchlings} practice a primordial, chaotic magic.  From a young 
age, they find themselves able to channel the latent energies of the 
world.  Misunderstood and often marginalized, Witchlings often live apart 
from society, separated by the blessing and curse of their abilities.

\item {\bf Wizards} are students of the arcane.  Often apprenticing at a 
young age, they spend years reading ancient tomes, learning spellcraft, 
brewing potions, and imbuing wands and staves with magical powers.  They 
are often seen in the company of some sort of familiar.
\end{itemize}

\subsection{Deities}

There are nine deities within the world of {\it Savage Lands}, collectively
referred to simply as, "the Nine".  Though the motives and actions of the
divine can never be neatly categorized by mortal men, it is believed that
three are largely good, three remain neutral, and three are deeply evil.

\subsubsection{The Good}

\begin{itemize}
\item Empress of the Heavens, {\bf Celeste} created the universe, crafting 
the galaxies and stars.  She watches the world from high above, where 
centuries pass like seconds.  Her domains are magic and creation.  
Throughout the existence of the universe, she has waged countless battles 
against the horrors of Sceadugenga.  Celeste grants her worshippers +1 
Intelligence.

\item Appearing to his followers as a great knight armoured in shining
plate and wielding a flaming blade, {\bf Aurelion} governs strength, 
chivalry, and honourable combat.  He is husband of The Lady, and is in 
constant struggle with the forces of black Urgoth.  Followers of Aurelion 
gain +1 Strength.

\item Wife of Aurelion, {\bf The Lady} is seen in the form of an 
impossibly beautiful and radiant woman.  She loves life, love, light, and 
music.  The Lady is the kindest and gentlest of the pantheon, and bestows 
+1 Charisma to her worshippers.
\end{itemize}

\subsubsection{The Neutral}

\begin{itemize}
\item After Celeste created the heavens, {\bf Vedere} created the world 
itself.  It was by his will that the mountains were raised.  The forests 
and grasslands are his, as are the lakes and seas, the deserts and 
marshes.  Itinerant wanderers and those who live off the land are often 
worshippers of Vedere, who protects his followers by granting +1 Health.

\item Once Vedere created the world, he made {\bf Voros} its protector.  
Voros lives deep within the molten core, a monstrous red wyrm, breathing 
gases and lava from his massive jaws.  Voros is often worshipped by the 
dwarves, and others who live deep within the earth.  Those who revere him 
receive +1 Strength.

\item Appearing as a cloaked figure accompanied by a crow, 
{\bf The Trickster} wanders the world of men, bending probability wherever 
he goes.  He acts as an agent of neutrality.  By his deeds, The Trickster 
balances the gains of order and chaos.  His blessing grants +1 Agility.
\end{itemize}

\subsubsection{The Evil}

\begin{itemize}
\item A withered, twisted figure, {\bf Shiver} takes the form of a bent and
haggard crone.  She comes in winter, on cold winds: her presence casts a 
pall over the landscape, a deathly chill that cannot be lifted.  To her 
followers, she grants +1 Willpower. 

\item The Black Ogre, {\bf Urgoth}, feeds off anger, rage, and hate.  He 
seeks the destruction of all things good and holy, and leads his hordes of 
chaos in an ongoing struggle against Aurelion's forces.  Those loyal to 
Urgoth receive +1 Strength.

\item A teeming black horror as old as time itself, {\bf Sceadugenga} lurks
in the blackest corners of the universe, held back by the power of the rest
of the pantheon.  He seeks nothing less than the destruction of all 
creation, culminating with the deaths of all the other gods.  His name is 
considered ill even to speak.  His few followers practice in utmost 
secrecy, and can be identified by a bleeding black mark on their 
foreheads.  Sceadugenga's minions receive +1 Intelligence.

\end{itemize}

\subsection{Statistics}

A character's statistics influence many things: its ability to land or
dodge blows, its hardiness, its ability to deal damage or learn spells,
and many other things.  Statistics are displayed at the bottom of the
screen

\subsubsection{Primary Statistics}

There are seven primary statistics: Strength, Dexterity, Agility, Health,
Intelligence, Willpower, and Charisma.  These have an impact on many
different calculations and outcomes within the game, and are described
in detail below.

\begin{itemize}
\item {\bf Strength} determines how strong a character is.  It affects
weapon damage, as well as the ability to hit with large weapons such as
great swords and huge hammers.

\item {\bf Dexterity} represents a character's physical control, and is
used to determine the ability to hit with most melee and ranged weapons.

\item {\bf Agility} is a character's nimbleness, and allows it to more
easily dodge attacks and other dangers.

\item {\bf Health} is a measure of a character's toughness or hardiness.
It affects how much damage the character can take before dying, as well
as helping to resist certain unwanted statuses.

\item {\bf Intelligence} affects the ability of a character to successfully
learn spells, and the number of spells that can be cast before exhaustion.

\item {\bf Willpower} also affects the ability of a character to learn
certain spells.  It also has an impact on the number of spells that the
character can cast before exhaustion, but to a lesser degree than
Intelligence.

\item {\bf Charisma} determines how easily the character sways or
influences others, and also impacts whether monsters are generated as
initially hostile or not.

\end{itemize}
\subsubsection{Secondary Statistics}

Secondary statistics are much more focused than primary statistics, and
tend to have much more specific uses.

\begin{itemize}
\item {\bf Evade} is used to determine whether a creature successfully
avoids an incoming attack.

\item {\bf Soak} reduces the amount of damage on a successful attack,
removing one point of damage per point of Soak.

\item {\bf Speed} determines how quickly a character can attack -- the
lower the Speed score, the sooner the character can act again.

\item {\bf Hit Points} represents a character's capacity for taking damage.
When a character's hit points are reduced to 0 or lower, the character is
dead.

\item {\bf Arcana Points} are a character's ability to cast spells.  Each
spell has a particular arcana point cost associated with it, so the higher
this score, the more spells the creature can cast before exhaustion.
\end{itemize}

\subsection{Skills}

\subsubsection{General}
\begin{itemize}
\item {\bf Archery}: Skill with ranged weapons.
\item {\bf Awareness}$\dagger$: General feelings; sixth sense.
\item {\bf Bargaining}$\dagger$: Lower prices in shops.
\item {\bf Beastmastery}$\dagger$: Tame wild creatures.
\item {\bf Blindfighting}$\dagger$: Fight effectively without sight. 
\item {\bf Boating}$\dagger$: Skill navigating on the water.
\item {\bf Bowyer}$\dagger$: Craft bows and crossbows.
\item {\bf Brewing}$\dagger$: Brew magic potions and moonshine. 
\item {\bf Cantrips}$\dagger$: Minor magical abilities. 
\item {\bf Carrying}$\dagger$: Carry more and heavier items.
\item {\bf Combat}: Skill with melee weapons. 
\item {\bf Crafting}$\dagger$: Skill at creating magical items. 
\item {\bf Desert Lore}$\dagger$: Knowledge of deserts and dunes. 
\item {\bf Detection}: Sense unseen things.
\item {\bf Disarm Traps}$\dagger$: Disassemble dangerous traps. 
\item {\bf Dual Wield}$\dagger$: Fight effectively with two weapons. 
\item {\bf Dungeoneering}$\dagger$: Ability to survive in dungeons.
\item {\bf Escape}: Flee effectively and squeeze through tight spaces.
\item {\bf Fishing}$\dagger$: Catch and salt fish.
\item {\bf Fletchery}$\dagger$: Create arrows and bolts. 
\item {\bf Foraging}$\dagger$: Find food in the wilderness.
\item {\bf Forest Lore}$\dagger$: Knowledge of the woods and forests. 
\item {\bf Herbalism}$\dagger$: Forage for useful herbs. 
\item {\bf Hiding}$\dagger$: Remain unseen from hostile foes.
\item {\bf Hunting}$\dagger$: Effectively hunt animals large and small. 
\item {\bf Intimidation}$\dagger$: Make your foes tremble and quiver. 
\item {\bf Jeweler}$\dagger$: Craft rings and amulets. 
\item {\bf Jumping}$\dagger$: Leap over obstacles and foes.
\item {\bf Leadership}$\dagger$: Command allies effectively in combat. 
\item {\bf Literacy}$\dagger$: Required to learn spells from tomes. 
\item {\bf Lore}$\dagger$: ???. 
\item {\bf Magic}$\dagger$: Learn spells more effectively. 
\item {\bf Marsh Lore}$\dagger$: Knowledge about swamps and fens. 
\item {\bf Medicine}$\dagger$: Naturally heal wounds faster. 
\item {\bf Mountain Lore}$\dagger$: Knowledge of mountain features. 
\item {\bf Mountaineering}: Skill at climbing tall peaks. 
\item {\bf Music}$\dagger$: Sway with songs and instruments. 
\item {\bf Night Sight}$\dagger$: Ability to see in low-light conditions. 
\item {\bf Oceanography}$\dagger$: Knowledge of seas and oceans. 
\item {\bf Papercraft}$\dagger$: Creation of paper from natural materials. 
\item {\bf Religion}$\dagger$: Knowledge of the Nine and their ways. 
\item {\bf Scribing}$\dagger$: Creation of scrolls and books. 
\item {\bf Skinning}$\dagger$: Make useful skins from dead creatures. 
\item {\bf Smithing}$\dagger$: Create and improve melee weapons.
\item {\bf Spelunking}$\dagger$: Move through tight spaces with ease.
\item {\bf Stealth}$\dagger$: Move about without being heard.
\item {\bf Swimming}: Swim through water without drowning.
\item {\bf Tanning}$\dagger$: Turn skins into useful armour.
\item {\bf Thievery}$\dagger$: Redistribute wealth towards yourself.
\item {\bf Weaving}$\dagger$: Create cloaks and clothing.
\end{itemize}

$\dagger$ -- Not implemented yet.

\subsubsection{Melee Weapons}

All melee weapons can be categorized as one of the following weapon skills: {\bf Axes}, {\bf Short Blades}, {\bf Long Blades}, {\bf Bludgeons}, {\bf Daggers}, {\bf Rods and Staves}, {\bf Spears}, {\bf Unarmed}, {\bf Whips}, {\bf Exotic}.

\subsubsection{Ranged Weapons}

All ranged weapons can be categorized as one of the following ranged weapon skills: {\bf Thrown Axes}, {\bf Thrown Blades}, {\bf Thrown Bludgeons}, {\bf Bows}, {\bf Crossbows}, {\bf Thrown Daggers}, {\bf Rocks}, {\bf Slings}, {\bf Thrown Spears}, {\bf Exotic Ranged}.

\subsubsection{Magic}

There are four spheres of magic, which encompass all the magics of the world.  These are:

\begin{itemize}
\item {\bf Arcane}:
\item {\bf Divine}:
\item {\bf Mystic}:
\item {\bf Primordial}:
\end{itemize}

\subsection{Resistances}

In {\it Savage Lands}, there are many different kinds of dangers.  Bandits
carry cheap knives and swords.  Ogres and trolls wield massive wooden
clubs.  Dragons breath infernal flames, and wizards cast ancient spells.
These are all represented by different damage types, and every creature
is affected differently, based on race, class, current equipment, and
the currently in-force spells.

All creatures in the game, not just the player's character, have a set of 
resistances.  Some creatures may be almost invincible when attacked with
certain damage types, and may be deeply vulnerable to others.

The different damage types follow, with an example of how they are used
in-game.

\begin{itemize}
\item {\bf Slash}: swords, axes, etc.
\item {\bf Pierce}: daggers, spears, etc.
\item {\bf Pound}: clubs, maces, etc.
\item {\bf Heat}: fire-based spells, dragon breath, fiery weapons, etc.
\item {\bf Cold}: ice-based spells, certain undead/astral creatures, etc.
\item {\bf Acid}: certain jellies and slimes, insects, demons, etc.
\item {\bf Poison}: certain snakes and reptiles, sea creatures, demons, etc.
\item {\bf Holy}: divine magics, holy creatures, etc.
\item {\bf Shadow}: primordial magics, certain undead/astral creatures, etc.
\item {\bf Arcane}: arcane magics, certain demons, etc.
\item {\bf Lightning}: lightning-based spells, certain holy creatures, etc.
\end{itemize}

The list above is intentionally incomplete -- there are many different
creatures scattered throughout the world, and a well-prepared character 
should have preparations in place for many different possibilities, not 
just the most common.

\end{document}

